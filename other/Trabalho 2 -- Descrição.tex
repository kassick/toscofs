% LaTex source code
% File: "Trabalho 2 -- Descrição.tex"
% Created: "Sex, 20 Nov 2015 21:50:45 -0200 (kassick)"
% Updated: "2017-10-26 17:07:37 kassick"
% $Id$
% Copyright (C) 2015, Rodrigo Virote Kassick <rvkassick@inf.ufrgs.br>

\documentclass[a4paper, oneside,12pt]{article}
\usepackage[portuguese]{babel}
\usepackage[]{hyphenat}
\usepackage{ifxetex}
\usepackage{listings}
\usepackage[%
        bookmarksnumbered=true,
        breaklinks=true,
        unicode=true,
        colorlinks=true,
        anchorcolor=black,
        citecolor=black,
        filecolor=black,
        linkcolor=black,
        urlcolor=black
]{hyperref}[2009/10/09]
\usepackage{enumitem}
\usepackage[sharp]{easylist}
\usepackage{multicol}
\usepackage{mathtools}
\usepackage{amsmath,amssymb,latexsym}         % simbolos, fontes, etc
\usepackage{wasysym}
%\usepackage{circuitikz}
%\usepackage{amsmath}
\usepackage{indentfirst}

\ifxetex
    \usepackage{tgtermes}
\else
    \usepackage[utf-8]{inputenc}
    \usepackage{times}              % pacote para usar fonte Adobe Times
    \usepackage[T1]{fontenc}
\fi

\usepackage[university={Uniritter -- Laureate International Universities$^{\noexpand\circledR}$},
            institute={Faculdade de Informática},
            courseid=INF0024,
            coursename={Sistemas Operacionais II},
            professorname={Rodrigo Kassick},
            %class={PNA},
            %semester={2015-2},
            border=false,
            gradingtable=false,
            sectionname={Exercício},
            description={Trabalho Prático},
            date={}]{unitest}
\begin{document}

\finishgrading    % Evita que apareça nome de aluno no topo da página

\section{Descrição do Trabalho} \label{sec:desc_prob}

É trabalho do sistema operacional gerenciar dispositivos de armazenamento e fornecer ao usuário uma noção conveniente -- como arquivos e pastas -- para trabalhar com seus programas.

A tarefa de um \textbf{sistema de arquivos} é justamente a de organizar algum meio de armazenamento em arquivos e diretórios.

O trabalho consiste em implementar um sistema de arquivos simples com diretório único e i-node de apenas um nível.


O sistema de arquivos a ser implementado fornecerá funções básicas (como open, read, write, etc.) e fará as alterações no disco simulado.

Todas as alterações no sistema de arquivos deverão ser refletidas no disco.


\section{Funções a Implementar}

As seguintes funções devem estar implementadas para o funcionamento do sistema de arquivos:

\begin{itemize}
    \item \texttt{mkfs} -- Cria um sistema de arquivos no disco (simulado). Esta função deverá escrever no disco o superblock, o bitmap e um diretório raiz vazio.

    \item \texttt{mount} -- Cria uma instância do sistema de arquivos a partir do conteúdo de um disco. Em algum momento, este disco teve o sistema de arquivos criado com mkfs. Porém, deve ser possível \emph{montar} o sistema de arquivos em execuções subsequentes do programa sem precisar recriá-lo.

        ``Montar'' significa que iremos carregar para a memória, pelo menos, a posição no disco onde se encontra o diretório raiz.

    \item \texttt{lsdir} -- Retorna um vetor ou lista com os nomes e o tamanho de cada arquivo do diretório.

    \item \texttt{create\_file(fname)} -- Cria um novo arquivo com o nome fornecido. Deverá ser refletido no disco a criação de um novo i-node para este arquivo, bem como a alteração no diretório.

        Lembre-se que, como um i-node ocupa espaço, o bloco deverá ser marcado como ocupado no bitmap, e a alteração no bitmap deverá ser refletida no disco.

    \item \texttt{delete\_file(fname)} -- Remove um arquivo. A entrada do diretório deverá ser apagada e todos os i-nodes associados com o arquivo deverão ser liberados.

    \item \texttt{open\_file(fname) -> file\_handle} -- Abre o arquivo com o nome fornecido e devolve um \emph{handle}. O handle pode ser uma classe \texttt{File} na qual deve ser possível chamar as funções de \emph{read} e \emph{write}.

        O handle deve estar associado ao i-node do arquivo (seja em memória ou em disco).

    \item \texttt{file\_write(handle, posicao, tamanho, bytes)} -- Escreve no arquivo aberto uma sequência de \texttt{tamanho} bytes posição \texttt{posicao}.

        Lembre-se que operações de escrita podem alterar o tamanho do arquivo. A operação também pode alterar o mapeamento dentro do i-node (por exemplo, foi escrito algo para um bloco pela primeira vez).

    \item \texttt{file\_read(handle, posicao, tamanho) -> bytes[]} -- Lê do arquivo aberto uma sequência de \texttt{tamanho} bytes a partir da posição \texttt{posicao}.

\end{itemize}




\section{Exemplo de Uso}

\begin{lstlisting}[language=C]
    disk_t * disk = disk_new("arquivo.dat");
    mkfs(disk);
    fs = mount(disk);
    create_file(fs, "arquivo1");
    int fd = open_file(fs, "arquivo1");
    file_write(fs, fd, 10, strlen("teste"), "teste");
\end{lstlisting}

\section{O Disco}
\label{sub:Disco}

O trabalho deverá utilizar um disco simulado (ver implementações de referência) capaz de fazer leitura e escritas de setores (e.g. 512 bytes), armazenando os dados em um arquivo ou outro meio de armazenamento.

O setor setor é um vetor de \textbf{bytes} de tamanho fixo (e.g. 512 ou 4096). Um disco contém uma quantidade fixa de setores, que são identificados por um número sequencial (i.e. setor 1, setor 2, ...)

São fornecidas funções de referência em C, C++ e Java para o disco.

As funções que o disco deve suportar são:

\begin{itemize}
    \item \texttt{new} -- Cria um novo \emph{objeto} disco utilizando um arquivo específico como armazenamento. E.g.:
        \begin{lstlisting}[language=c]
            // Novo disco simulado com 2000 setores
            disk_t * disk = disk_new("arquivo.dat", 2000);
        \end{lstlisting}

        ou

        \begin{lstlisting}[language=java]
            // Novo disco simulado com 2000 setores
            Disk d = new Disk("arquivo.dat", 2000);
        \end{lstlisting}

    \item \texttt{write\_sector} -- Escreve o conteúdo de um buffer da memória em um \textbf{setor} (i.e. vetor de tamanho fixo) no disco. O buffer de memória tem que ser to exato tamanho do setor do disco. E.g.:

        \begin{lstlisting}[language=c]
            char conteudo_setor[SECTOR_SIZE];
            // preenche setor
            // Escreve o conteudo no setor de nro 5
            disk_write(disk, 5, conteudo_setor);
        \end{lstlisting}

        ou

        \begin{lstlisting}[language=java]
            byte[] data = new byte[Disk.SECTOR_SIZE];
            // preenche data
            d.write_sector(1, data);
        \end{lstlisting}

    \item \texttt{read\_sector} -- Lê o conteúdo de um setor do disco para a memória. E.g.

        \begin{lstlisting}[language=c]
            char conteudo_setor[SECTOR_SIZE];
            // Escreve o conteudo no setor de nro 5
            disk_read(disk, 5, conteudo_setor);
            // Usa conteudo_setor
        \end{lstlisting}

        ou

        \begin{lstlisting}[language=java]
            byte[] data = new byte[Disk.SECTOR_SIZE];
            d.read_sector(1, data);
            // Decodifica data
        \end{lstlisting}

\end{itemize}

\section{O Diretório}

O sistema de arquivos a ser desenvolvido necessita ter um único diretório -- o diretório raiz. Todos os arquivos são criados neste diretório.

O diretório raiz deve ser capaz de conter \emph{pelo menos} 32 nomes de arquivos (i.e. poderemos criar até 32 arquivos). O tamanho do nome de arquivo e a maneira de organizar os nomes de arquivos dentro dos blocos é livre. Sugere-se uma estrutura de tabela simples (vetor de struct, com nome de tamanho máximo definido).

Cada nome de arquivo deste diretório apontar para um \emph{i-node} no disco. A informação de mapeamento dos blocos do arquivo para os setores do disco devem estar no \emph{i-node}, não na entrada de diretório do arquivo.

\section{I-Node}

O \emph{i-node} do sistema de arquivos deve fornecer o mapeamento entre os blocos do arquivo e os setores do disco que armazenam os blocos.

Isto é, ele deverá ser alguma estrutura na qual o acesso à posição \texttt{n} indique onde no disco está guardado o n-ésimo bloco do arquivo.
\begin{lstlisting}[language=C]
    pritnf("O bloco %d está guardado na posição %d\n",
                     n,                         inode->mapping[n]);
\end{lstlisting}

Esta tabela pode ser de apenas um nível (i.e. sem ponteiros para outros blocos com ponteiros para setores). Caso seja utilizada esta estratégia, lembre-se que o tamanho do arquivo estará limitado a um número pequeno (e.g. 2MB).

Estratégias com blocos de continuação e endereçamento de 2 ou 3 níveis, se feitas corretamente, podem garantir pontos extras.


\section{Gerenciamento de espaço livre}
\label{sub:gerenciamento_de_espaco_livre}

O sistema de arquivos deve gerenciar quais blocos estão livres tanto para gravar estruturas do sistema de arquivos quanto os próprios dados dos arquivos (conteúdo).

Sugere-se utilizar uma estrutura de \emph{bitmap} como vista em aula. Uma implementação de referência está disponível em linguagem C. Linguagens como Java, C++ e C\# fornecem estruturas de dados de bitmap que se comportam como vetores de valores booleanos.

Lembre-se que esta estrutura deverá, posteriormente, ser convertida em setores, que são vetores de bytes. Este processo (serialização) é dependente da linguagem de programação utilizada.

\begin{itemize}
    \item Java: java.util.BitSet ; Utilize a função toBytes() para serializar a estrutura em uma sequência de bytes que poderá, depois, ser gravada em setores do disco.

    \item C++: {\tt std::vector<bool>} . Para fazer a serialização, utilize o código a seguir\footnote{\url{http://stackoverflow.com/questions/4666450/how-does-one-store-a-vectorbool-or-a-bitset-into-a-file-but-bit-wise}}:
        \begin{lstlisting}[language=C++]
            std::vector<bool> data(nblocks);
            /* obtain bits somehow */

            // Reserve an appropriate number of byte-sized buckets.
            std::vector<char> bytes((int)std::ceil((float)data.size() / CHAR_BITS));

            for(int byteIndex = 0; byteIndex < bytes.size(); ++byteIndex) {
               for(int bitIndex = 0; bitIndex < CHAR_BITS; ++bitIndex) {
                   int bit = data[byteIndex * CHAR_BITS + bitIndex];

                   bytes[byteIndex] |= bit << bitIndex;
               }
            }
        \end{lstlisting}

    \item C\#: System.Collections.BitArray . Para serializar, utilize o BinaryFormatter\footnote{Ver \url{https://social.msdn.microsoft.com/Forums/vstudio/en-US/650bd86a-3f32-4629-921c-8baf20754505/save-bitarray-type-to-binary-file}}.

        \begin{lstlisting}[language={[Sharp]C}, showspaces=false ]
            BitArray occupied = new BitArray(nblocks, false);
            // ... preenche occupied
            Byte[] bytes = new Byte[Math.Ceiling(nblocks/8.0)];
            MemoryStream ms = MemoryStream(bytes);
            BinaryFormatter BF = new BinaryFormatter();
            BF.serialize(ms, occupied);
            // bytes contem os dados a serem gravados nos setores
        \end{lstlisting}

\end{itemize}

Lembre-se que as estruturas do sistema de arquivos ocupam espaço em disco, e portanto o bitmap deve saber que os blocos utilizados por estas estruturas estarão ocupados (i.e. diretório, i-nodes, superblock, o próprio bitmap, além dos blocos que guardam dados dos arquivos).

Quando algum novo bloco precisar ser ocupado no sistema de arquivos, localize uma posição no bitmap que não esteja marcada como ocupada, altere o bitmap e utilize a posição como número do bloco que estava livre.

Sempre que um novo arquivo for criado, pelo menos um bloco será ocupado por seu i-node. Os blocos de dados do arquivo podem ser alocados apenas na primeira vez que alguma posição daquele bloco for escrita (alocação sob demanda).

\section{Controle de tamanho de arquivo}
\label{sub:controle_de_tamanho_de_arquivo}

Cada arquivo deve ter pelo menos uma propriedade -- o seu tamanho, em bytes. O tamanho de um arquivo corresponde à última posição dentro do arquivo escrita -- i.e. em um arquivo recém criado cuja primeira escrita é no byte de nro 100, o tamanho passa a ser 100.

Um arquivo de 1MBytes possuirá 244 blocos de 4KB. Porém, nem todos esses blocos precisam estar alocados para o arquivo, apenas os blocos que tiveram seus bytes gravados (alocação sob demanda).

\end{document}
